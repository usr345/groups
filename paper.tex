\documentclass[12pt]{article}
\usepackage[a4paper]{geometry}
\geometry{verbose,tmargin=2cm,bmargin=2cm,lmargin=2cm,rmargin=2cm}
\usepackage[T2A]{fontenc}
\usepackage[english,russian]{babel}
\babelfont{rm}{CMU Concrete}
\babelfont{sf}{CMU Sans Serif}
\babelfont{tt}{CMU Typewriter Text}

\usepackage{graphicx}
\usepackage{float}
\usepackage{minted}
\usepackage{amsmath, amsthm, amssymb, amsfonts}
\newcounter{thm}
\newtheorem{proposition}[thm]{Предложение}
\title{\vspace{-5ex}Формализация теории групп на базе книги Алексеева В.Б. ``Теорема Абеля в задачах и решениях''\vspace{-4ex}}
\begin{document}
\maketitle

Прежде всего дадим формальные определения понятям ``полугруппа'', ``моноид'' и ``группа''.

\begin{minted}[breaklines, breakanywhere, tabsize=2]{coq}
(* G - носитель группы *)
(* mult - групповая операция *)
Class Semigroup G : Type :=
{
    mult : G -> G -> G;
    assoc : forall x y z:G,
        mult x (mult y z) = mult (mult x y) z
}.

Class Monoid G `{Hsemi: Semigroup G} : Type :=
{
  e : G;
  left_id : forall x: G, mult e x = x;
}.

Class Group G `{Hmono: Monoid G} : Type :=
{
    inv : G -> G;
    left_inv : forall x: G, mult (inv x) x = e;
}.
\end{minted}

Зададим нотацию для операции умножения в групповом контексте:

\begin{minted}[breaklines, breakanywhere, tabsize=2]{coq}
Declare Scope group_scope.
Infix "*" := mult (at level 40, left associativity) : group_scope.
\end{minted}

Мы начнем с задачи № 24 из книги.

\begin{proposition}
Пусть $a, b$ — произвольные элементы некоторой группы $G$. Доказать, что каждое из уравнений $a \cdot x = b$ и $y \cdot a = b$ имеет, и притом ровно одно, решение в данной группе.
\end{proposition}

Сначала рассмотрии первое уравнение, и докажем единственность при умножении на $a$ слева.

\begin{minted}[breaklines, breakanywhere, tabsize=2]{coq}
Open Scope group_scope.
Section Group_theorems.

  Variable G: Type.
  Context `{Hgr: Group G}.

  Proposition left_cancel : forall a x y:G,
      a * x = a * y -> x = y.
  Proof.
    intros a x y H. assert (inv a * (a * x) = inv a * (a * y))
      as Hinvx.
    - rewrite H. reflexivity.
    - repeat rewrite assoc in Hinvx. rewrite left_inv in Hinvx. repeat rewrite left_id in Hinvx. assumption.
  Qed.

  Proposition left_cancel_alt_proof : forall a x y:G,
      a * x = a * y -> x = y.
  Proof.
    intros a x y H.
    (* Здесь самое интересное: по сути это умножение гипотезы слева на inv a, заменяющее assert в предыдущей теореме *)
    (* f_equal : forall (A B : Type) (f : A -> B) (x y : A), *)
    (*    x = y -> f x = f y *)
    apply f_equal with (f := fun g:G => inv a * g) in H.
    (* Дальше всё одинаково *)
    repeat rewrite assoc in H. repeat rewrite left_inv in H. repeat rewrite left_id in H. assumption.
  Qed.
\end{minted}

Для доказательства второго предложения: единственности решения при умножении справа $y \cdot a = b$, нам понадобится доказать 2 вспомогательных предложения.

\begin{proposition}
  $\forall x \in G, x \cdot e = x$
\end{proposition}

\begin{minted}[breaklines, breakanywhere, tabsize=2]{coq}
  Proposition right_id : forall x:G, x * e = x.
  Proof.
    intro. apply left_cancel with (a := inv x). rewrite assoc. repeat rewrite left_inv. rewrite left_id. reflexivity.
  Qed.
\end{minted}

\begin{minipage}{.46\textwidth}
\mintinline{coq}{Proof.}
\begin{verbatim}
G : Type
Hsemi : Semigroup G
Hmono : Monoid G
Hgr : Group G
------------------------
forall x : G, x * e = x
\end{verbatim}
\end{minipage}
\hfill
\begin{minipage}{.46\textwidth}
\mintinline{coq}{intro.}
\begin{verbatim}
G : Type
Hsemi : Semigroup G
Hmono : Monoid G
Hgr : Group G
x : G
---------------------
x * e = x
\end{verbatim}
\end{minipage}

\bigskip
\noindent\makebox[\textwidth]{\hrulefill}

\bigskip
\begin{minipage}{.46\textwidth}
\mintinline{coq}{apply left_cancel with (a := inv x).}
\begin{verbatim}
G : Type
Hsemi : Semigroup G
Hmono : Monoid G
Hgr : Group G
x : G
----------------------------
inv x * (x * e) = inv x * x
\end{verbatim}
\end{minipage}
\hfill
\begin{minipage}{.46\textwidth}
\mintinline{coq}{rewrite assoc.}
\begin{verbatim}
G : Type
Hsemi : Semigroup G
Hmono : Monoid G
Hgr : Group G
x : G
-------------------------
inv x * x * e = inv x * x
\end{verbatim}
\end{minipage}

\bigskip
\noindent\makebox[\textwidth]{\hrulefill}

\bigskip
\begin{minipage}{.46\textwidth}
\mintinline{coq}{repeat rewrite left_inv.}
\begin{verbatim}
G : Type
Hsemi : Semigroup G
Hmono : Monoid G
Hgr : Group G
x : G
-------------------
e * e = e
\end{verbatim}
\end{minipage}
\hfill
\begin{minipage}{.46\textwidth}
\mintinline{coq}{rewrite left_id. reflexivity.}
\begin{verbatim}
G : Type
Hsemi : Semigroup G
Hmono : Monoid G
Hgr : Group G
x : G
--------------------
e = e
\end{verbatim}$\square$
\end{minipage}

\begin{proposition}
$\forall x \in G, x \cdot x^{-1} = e$
\end{proposition}

\begin{minted}[breaklines, breakanywhere, tabsize=2]{coq}
  Proposition right_inv:
    forall x:G, x * inv x = e.
  Proof.
    intro. apply left_cancel with (x := inv x). rewrite assoc.
    rewrite left_inv. rewrite left_id. rewrite right_id. reflexivity.
  Qed.
\end{minted}

\begin{proposition}
$\forall x \in G, x \cdot x^{-1} = e$
\end{proposition}
\end{document}
