\documentclass[12pt]{article}
\usepackage[a4paper]{geometry}
\geometry{verbose,tmargin=2cm,bmargin=2cm,lmargin=2cm,rmargin=2cm}
\usepackage[T2A]{fontenc}
\usepackage[english,russian]{babel}
\usepackage{graphicx}
\usepackage{float}
\usepackage{amsmath, amsthm, amssymb, amsfonts}
\begin{document}

Симметрия фигуры --- это преобразование фигуры саму в себя, сохраняющее расстояние между всеми её точками. Рассмотрим равносторонний треугольник, и одну из симметрий --- вращение вокруг его центра. Вращениям правильного треугольника соответствуют следующие подстановки:

\begin{figure}[H]
  \centering
  \begin{minipage}{.2\textwidth}
    \centering
    \includegraphics{./images/triangle.pdf}
  \end{minipage}
  \begin{minipage}{.2\textwidth}
    \centering
    $
    \left(\begin{array}{ccc}
      A & B & C \\
      A & B & C\\
    \end{array}\right)
    $

    \hrulefill\\
    e = 0
  \end{minipage}
  \;
  \begin{minipage}{.2\textwidth}
    \centering
    $
    \left(\begin{array}{ccc}
      A & B & C \\
      B & C & A\\
    \end{array}\right)
    $

    \hrulefill\\
    a = 1
  \end{minipage}
  \;
  \begin{minipage}{.2\textwidth}
    \centering
    $
    \left(\begin{array}{ccc}
      A & B & C \\
      C & A & B
    \end{array}\right)
    $

    \hrulefill\\
    b = 2
  \end{minipage}
\end{figure}

Таблица умножения поворотов треугольника приведена ниже. Заметим, что она эквивалентна таблице умножения $\mathbb{Z}_3$.

\begin{table}[H]
  \centering
  \begin{tabular}{ |c|*{3}{c}| }
    \hline
    & e & a & b\\
    \hline
    e & e & a & b\\
    a & a & b & e\\
    b & b & e & a\\
    \hline
  \end{tabular}
  $\Longleftrightarrow$
  \begin{tabular}{ |c|*{3}{c}| }
    \hline
    & 0 & 1 & 2\\
    \hline
    0 & 0 & 1 & 2\\
    1 & 1 & 2 & 0\\
    2 & 2 & 0 & 1\\
    \hline
  \end{tabular}
\end{table}

Кроме вращения у равностороннего треугольника есть еще 3 симметрии: отражения относительно осей $l_1, l_2, l_3$.

\begin{figure}[H]
  \centering
  \begin{minipage}{.5\textwidth}
    \centering
    \includegraphics{./images/triangle-line1.pdf}
  \end{minipage}
  \begin{minipage}{.2\textwidth}
    \centering
    $
    \left(\begin{array}{ccc}
      A & B & C \\
      A & C & B\\
    \end{array}\right)
    $

    \hrulefill\\
    c
  \end{minipage}
\end{figure}
\begin{figure}[H]
  \centering
  \begin{minipage}{.5\textwidth}
    \centering
    \includegraphics{./images/triangle-line2.pdf}
  \end{minipage}
  \begin{minipage}{.2\textwidth}
    \centering
    $
    \left(\begin{array}{ccc}
      A & B & C\\
      C & B & A\\
    \end{array}\right)
    $

    \hrulefill\\
    d
  \end{minipage}
\end{figure}

\begin{figure}[H]
  \centering
  \begin{minipage}{.5\textwidth}
    \centering
    \includegraphics{./images/triangle-line3.pdf}
  \end{minipage}
  \begin{minipage}{.2\textwidth}
    \centering
    $
    \left(\begin{array}{ccc}
      A & B & C\\
      B & A & C\\
    \end{array}\right)
    $

    \hrulefill\\
    f
  \end{minipage}
\end{figure}

Помпозиция преобразований будет соответствовать перемножению соответствующих подстановок. Составим программу для генерации таблицы умножения.

\begin{center}
Теорема 1
\end{center}

Доказать, что в произвольной группе $e^{-1} = e$.

\begin{center}
Доказательство
\end{center}

Аксиомы группы:

\noindent
$A_1: \forall a \in A, e \times a = a \times e = a$\\
$A_2: \forall a \in A, \exists a^{-1} \in A, a \times a = a^{-1} \times a = e$

Равенство $e^{-1} = e$ следует из последовательного применения, e к $A_2$ и $A_1$:

$$
e \times e^{-1} = e \implies e^{-1} = e
\qed
$$
\begin{center}
Теорема 2
\end{center}
Доказать, что $a^{-1} = (a^{-1})^{-1}$.

\begin{center}
Доказательство
\end{center}

Мы знаем, что $\forall a \in A, a \times a^{-1} = e$. Следовательно: $a^{-1} \times (a^{-1})^{-1} = e$. Домножим слева обе части равенства на a:
$$
a \times (a^{-1} \times (a^{-1})^{-1}) = a \times e \implies
(a \times a^{-1}) \times (a^{-1})^{-1}) = a \implies
e \times (a^{-1})^{-1}) = a
\implies (a^{-1})^{-1} = a
\qed
$$
\end{document}
